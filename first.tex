\documentclass[a4paper]{article}
\usepackage[UTF8]{ctex}
\usepackage[round]{natbib}
\renewcommand{\bibname}{reference}
\usepackage{titlesec}
%\usepackage{cite}
\usepackage{graphicx}
\usepackage{caption}
\usepackage{float}
\usepackage{graphicx, subfig}
\usepackage{amsmath}
\usepackage[colorlinks,linkcolor=blue]{hyperref}
\newtheorem{theorem}{Theorem}[section]
\newtheorem{lemma}{Lemma}[section]
\newtheorem{proof}{Proof}[section]
\newtheorem{defn}{Definition}[section]
\newtheorem{conj}{Conjecture}[section]
\newtheorem{exmp}{Example}[section]
\usepackage{hyperref}
\author{李国鹏}
\title{Hoe does information affect people's perception }
\date{}
\begin{document}
\maketitle
\tableofcontents
\section{Introduction}
近期,新冠肺炎给中国社会经济带来了巨大的损失。在新冠肺炎疫情发展的过程中,信息的传播给人们的心理产生了影响,造成了不同程度的恐慌心理。一定程度的恐慌是好事,有助于我个体加强防范,从而更早的战胜疫情,但是过度的恐慌导致了很多地方盲目断路、恐惧“武汉人”、“湖北人”,也影响了春节复工的节奏。因此,分析信息如何影响个体的恐慌程度十分重要。不管是信息的传播还是恐慌心理的蔓延,都离不开人际网络,因此本文尝试用网络经济学的方法研究在人际网络的作用下,疫情信息的传播会对社会产生什么样的影响。本文用到的一个主要的工具是 Friendship Paradox 。
不能有中文?


\section{Model}
\subsection{Setting}
首先假设存在一个社会网络(network) $(N,g)$,$N=\{1,...,n\}$ 表示社会中的成员(players), $g$是一个$n\times n$的矩阵,刻画了网络结构,如果$ij$是朋友,那么$g_{ij}=g_{ji}=1$。($g_{ii}=0,\,\,\forall i\in N$)$i$和$j$是社会网络里面的成员,$i$ 的degree$d_i(g)$ 表示player $i$ 的朋友的数量。每个人都只知道自己的degree但并不知晓整个网络的结构。$  P(d_i) $ 刻画了个体 $i$ 的degree 的 distribution。$E_i[d]$ 是degree的期望,$ Var_i[d]$是degree的方差。假设t时刻网络中有$m_t(m_t\leq n)$个个体被传染了。假设如果t时刻node $i$ 和被传染的个体$j$是邻居,则$t+1$ 时刻$i$被$j$传染的概率是$\alpha\,(\alpha\in [0,1])$ \footnote{当$\alpha=1$时,规定$(1-\alpha)^0=0^0=1$}。
\subsection{不考虑网络结构}
如果不考虑网络结构。如果个体 $i$ 在$t$ 时刻是健康的,并且个体 $i$ 和其他个体接触的概率是等同的,并且接触不同个体的概率是独立的。那么$t+1$ 时刻 $i$ 被传染的概率为:
$ \frac{1}{n-1}*\alpha*m_t=\frac{m_t\alpha}{n-1}  $ 。

\subsection{考虑网络结构之后 }

如果个体$i$的degree为$d_i$,并且$t$时刻$i$是健康的,那么$t+1$时刻 $i$ 被感染的概率$\hat{P}_{i,t+1}(d_i)$为:

\[\hat{P}_{i,t+1}(d_i)  =  1-[\frac{C_{m_t}^0*C_{n-1-m_t}^{d_i}} {C_{n-1}^{d_i}} +\frac{C_{m_t}^{1}*C_{n-1-m_t}^{d_i-1}} {C_{n-1}^{d_i}}(1-\alpha)^{1} \]
\[+  ...+  \frac{C_{m_t}^{m_t}*C_{n-1-m_t}^{d_i-m_t}}{C_{n-1}^{d_i}}(1-\alpha)^{m_t}    ]=1-\sum_{k=0}^{m_t} \frac{C_{m_t}^{k}*C_{n-1-m_t}^{d_i-k}}  {C_{n-1}^{d_i}} (1-\alpha)^k \]

( 当 $d_i\leq n-1-m_t$  并且 $d_i \geq m_t$ 时)
\paragraph{第二种情况}
\[\hat{P}_{i,t+1}(d_i)  =  1-[\frac{C_{m_t}^0*C_{n-1-m_t}^{d_i}} {C_{n-1}^{d_i}} + ...+  \frac{C_{m_t}^{d_i}*C_{n-1-m_t}^{0}}{C_{n-1}^{d_i}}(1-\alpha)^{d_i} ]   \]

\[=1-\sum_{k=0}^{d_i} \frac{C_{m_t}^{k}*C_{n-1-m_t}^{d_i-k}}  {C_{n-1}^{d_i}} (1-\alpha)^k\]



( 当 $d_i\leq n-1-m_t$  并且 $d_i < m_t$ 时)



\[\hat{P}_{i,t+1}(d_i) =1-[\frac{C_{n-1-m_t}^{0 } C_{m_t}^{  d_i}        } {C_{n-1}^{d_i}} (1-\alpha)^{d_i} \]
\[ +...+
\frac{C_{n-1-m_t}^{n-1-m_t } C_{m_t}^{  d_i-(n-1-m_t)}        } {C_{n-1}^{d_i}} (1-\alpha)^{d_i-(n-1-m_t)}  ]\]
\[=1-\sum_{k=0}^{n-1-m_t}\frac{C_{n-1-m_t}^kC_{m_t}^{d_i-k} }{C_{n-1}^{d_i}}
(1-\alpha)^{d_i-k}\]

( 当 $d_i > n-1-m_t$  并且 $d_i < m_t$ 时,此处思路:从邻居中最多有多少人健康——$(n-1-m_t)$,到邻居中最少有多少人健康——$(0)$ )

\[\hat{P}_{i,t+1}(d_i) =1-[\frac{C_{n-1-m_t}^{d_i-m_t }  C_{m_t}^{  m_t}        } {C_{n-1}^{d_i}} (1-\alpha)^{m_t} \]

\[+...+ \frac{C_{n-1-m_t}^{n-1-m_t } C_{m_t}^{  d_i-(n-1-m_t)}        } {C_{n-1}^{d_i}} (1-\alpha)^{d_i-(n-1-m_t)}  ]\]

\[  =1-\sum_{k=d_i-m_t}^{n-1-m_t}\frac{C_{n-1-m_t}^{k}C_{m_t}^{d_i-k} }{C_{n-1}^{d_i}}
(1-\alpha)^{d_i-k}\]
( 当 $d_i > n-1-m_t$  并且 $d_i \geq m_t$ 时,此处思路:从邻居中最多有多少人健康——$(n-1-m_t)$,到邻居中最少有多少人健康——$(d_i-m_t)$ )

\paragraph{}
平均而言个体的degree为$E_i(d)$ \footnote{ 当然,计算组合数$C_n^m$ 时$m$ 要求为整数,因此我们按照四舍五入方式对$E_i(d)$进行取整},可以将$E_i(d)$ 带入上述公式,求出网络中的个体平均被传染的概率:
\[  \bar{P}_{i,t+1}(d) \]

\paragraph{}
在个体计算出自己被传染的概率之后,会产生一定程度的恐慌,然后会通过社交媒体、聊天等方式显示出来,此处表现为传导给自己的邻居。因此,每个个体也会看到自己邻居的患病的概率——除了邻居之外不能得知其他人的。由于个体只拥有局部信息,因此他不知道自己处于网络的什么位置。每个个体都会受到邻居们对自己的影响。会根据邻居的患病概率更新自己的患病概率——将邻居患病的均值作为自己的更新之后的患病概率,或者说如果邻居都比自己高,则自己会将之前估计的患病的概率提高一些,提高到和邻居的均值一样的水平,反之亦然。平均而言邻居的degree的为$\tilde{E_i}[d]=$。
平均而言邻居患病的概率为:
\[\hat{\tilde{P}}_{i,t+1}    \]

\paragraph{}
因此平均而言个体会认为$t+1$ 时刻自己的患病概率为:$\hat{\tilde{P}}_{i,t+1}$,实际上平均而言个体的患病概率为:$ \bar{P}_{i,t+1}(d)$。
\paragraph{}
\[ \tilde{E_i}[d]=E_i(d)+\frac{Var_i(d)  }{E_i(d)} \]
\paragraph{传染性很强时——极端情况:$\alpha=1$ 时 }
\[\hat{\tilde{P}}_{i,t+1}-  \bar{P}_{i,t+1}(d)=   1-\frac{C_{n-1-m_t}^{\tilde{E_i}[d]}} {C_{n-1}^{\tilde{E_i}[d]}} -( 1-\frac{C_{n-1-m_t}^{E_i(d)}} {C_{n-1}^{E_i(d)}}  )      \]

\[  =  \frac{C_{n-1-m_t}^{E_i(d)}} {C_{n-1}^{E_i(d)}}    -\frac{C_{n-1-m_t}^{\tilde{E_i}[d]}}{C_{n-1}^{\tilde{E_i}[d]}} \]

\[ =\frac{(n-1-m_t)*...*(n-1-m_t-E_i[d]+1)}{(n-1)*...*(n-1-E_i[d]+1)}   \] \[-\frac{(n-1-m_t)*...*(n-1-m_t-\tilde{E_i}[d]+1)}{(n-1)*...*(n-1-\tilde{E_i}[d]+1)}    \]

\[= \frac{(n-1-m_t)*...*(n-1-m_t-E_i[d]+1)}{(n-1)*...*(n-1-E_i[d]+1)}\]
\[ * [    1-   \frac{(n-1-m_t-E_i[d])*...*(n-1-m_t-\tilde{E_i}[d]+1  )  }{(n-1-E_i[d])*...*(n-1-\tilde{E_i}[d]+1  )}    ]  > 0  \]
\paragraph{}
此时,平均而言,个体会高估自己被感染的概率,并且方差越大,高估的程度越高。
\paragraph{现实场景}
政府公布了被感染人数,然后个体看到后纷纷根据自己的朋友的数量计算出$\hat{P_i}(d_i) $,然后通过社交媒体被自己的朋友观察到。当然,自己也可以观察到自己的朋友,然后大家纷纷根据观察到的朋友的被感染概率更新自己被传染的概率。


\subsection{进行比较}
假设:
\[ \frac{C_{m_t}^{k}*C_{n-1-m_t}^{d_i-k}}  {C_{n-1}^{d_i}}(1-\alpha)^k=C_{m_t}^{k}(1-\alpha)^k*\frac{(n-1-m_t)*...*[n-1-m_t-(d_i-k)+1]}{(d_i-k)!} \]
\[*\frac{d_i!}{(n-1)*...*(n-1-d_i+1)}\]
为了分析预期被传染概率随着$d_i$的变化情况,不妨设
\[f(d_i)= \frac{(n-1-m_t)*...*[n-m_t-(d_i-k)]}{(d_i-k)!}* \frac{d_i!}{(n-1)*...*(n-d_i)}\]

比较\[\frac{f(d_i+l)}{f(d_i)}=\frac{(d_i+l)*...*(d_i+1)}{(d_i-k+l)*...(d_i-k+1)}  *
\frac{(n-m_t-d_i+k-1)*...*(n-m_t-d_i+k-l)}{(n-d_i-1)*...*(n-d_i-l)}     \]


\subsection{动态}
现在进入$t+2$ 时刻,此时个体需要更新自己被传染的概率,此时他知道网络中平均而言有
$ m_{t+1}=(n-m_t)*\bar{P}_{i,t+1}(d)  +m_t $,依然计算出自己被传染的概率,然后依据朋友的信息更新自己被传染的概率,一直循环下去长此以往,直到$m_{t+1}=n$!

\section{Small World Networks}
Traditionally, the connection topology is assumed to be either completely regular or completely random.
This model makes regular networks ‘rewired’ to introduce
increasing amounts of disorder.
\paragraph{}
Starting from a ring lattice with $n$ nodes and $k$ edges per vertex, we rewire each edge at random with probability$p$. That is to say : every will be deleted with a probability $p$, and relinked to any two random nodes.(比如一共有$\frac{n(n-1)}{2}$ pairs of nodes, we select one pair randomly.



\subsubsection{Characteristics of Small World Networks}
\begin{itemize}
	\item highly clustered
	\item small characteristic path lengths——diameter
	
\end{itemize}
Starting from a ring lattice with $n$ vertices and $k$ edges per nodes, we rewire each edge at random with probability $p$.

\begin{itemize}
	\item $L(p)$  :  characteristic path length——diameter, which provides whole information
	\item $C(p)$  :  clustering coefficient, which provides local information
\end{itemize}

\[n \gg k \gg ln(n) \gg 1\]
$k\gg ln(n)$  guarantees that a random graph will be connected.
在这个范围内,可以知道:
\begin{itemize}
	\item 当$p \rightarrow 0$ 时,可以得到:$L\sim n/2k \gg 1$  并且 $C\sim 3/4$
	\item  当$p \rightarrow 1$ 时,可以得到:$L\approx L_{random} \sim ln(n)/ln(k)$ 并且 $C\approx C_{random} \sim k/n \gg 1$
	
\end{itemize}

These limiting cases might lead one to suspect that large C is always associated with large L, and small C with small L.
there is a broad interval of p
over which $L(p)$  is almost as small as $L_{random}$ yet
$C(p) \gg C_{random}$
\subsection{动态观点}

在$t=0$ 时刻,一个被感染的人被引入了一个由健康人群构成的网络,
在一个患病周期,这个人要么死了,要么被治好了,不管是哪一种情况均不再具有传染力,也没有被传染的可能性——因此被移出该网络。
在这个周期内,它可以以$r$ 的概率感染它的每一个邻居。这个过程会一直进行,直到——全部被感染了,感染了一些人之后最终消失了。


\section{Power-law Network}
A common property of many large networks is that the vertex connectivities follow a scale-free power-law
distribution. Because :

\begin{itemize}
	\item Networks expand continuously by the addition of new vertices, and
	\item New vertices attach preferentially to sites that are already well connected.
\end{itemize}
\paragraph{}
已有网络模型的缺点:
existing network models fail to incorporate growth and preferential attachment
\paragraph{}

\[P(k)\sim k^{-\gamma}\]

之前的两个网络都认为当$k$很大的时候,$P(k)$会非常小,而此处的模型认为不是
这样子的$P(k)$ 也会相对比较大。
\paragraph{}
一开始有$M_0={1,2,...m_0}$ ($m_0$ 是一个比较小的数)个nodes,然后每一个time step
我们添加一个新的$node$,并且有$m(m\leq m_0)$ links和已经存在的$m_0$ 中的
$m$个links相连,并且为了引入preferential attachment,我们让新的node和已经存在的 node $i\in M_0$ 连接的概率是$\prod(k_i)= k_i/{\sum_{j} k_j}$。
在$t$个time steps之后,这个模型会产生一个随机网络,里面有$t+m_0$个nodes和
$mt$ 个edges。这个网络会演化为一个$scale-invariant$ 的状态。某个node有$k$个
edges的概率服从power law,并且$\gamma_{model}=2.9 \pm 0.1$。
\begin{itemize}
	\item 初始的网络是怎么样的?是互相连接的吗?如果完全没有连接,那怎么办呢?分母就是0了
	\item 为什么每次都会有$m$个links生成?因为引入一个新的nodes,即使没有第一个问题,那和
	每个node产生link的概率也是给定的,为什么一定会有$m$ 个links生成呢?
	\item $P(k)$ 和时间 $t$ 无关
\end{itemize}
\subsection{变形——去掉preferential attachment}
\[\prod (k)=const=\frac{1}{m_0+t-1}\]
因为第一次的概率是$\frac{1}{m_0}$,因此第$t$ 次的概率是$\frac{1}{m_0+t-1}$。
此时:
\[  P(k)\sim e^{-\beta k} \]
\subsection{变形——去掉growing}
we start with $N$ nodes and no edges. 然后每一阶段,随机选择一个node,然后把它和
任意一个其他的node$i$ 以$\prod (k_i)=k_i/\sum_{j}k_j$ 的概率相连。此时,尽管依然呈现power law,但是此时$P(k)$不稳定。因为$N$是固定的,而edges的数量在一直增长,大概到$T \simeq N^2$ 时,所有的nodes都是connected。
\subsection{richer-get-richer}

The rate at which a vertex acquires edges :
\[\frac{\partial k_i}{\partial t}=\frac{k_i}{2t}\]
which gives:
\[k_i(t)=m(t/t_i)^{0.5}\]
where $t_i$ is the time at which node $i$ was added to the system.
\paragraph{}
The probability that a vertex $i $ has a connectivity smaller than $k$,
$P(k_i(t)<k)$, can be written as $P(t_i>m^2t/k^2)=1-P( t_i \leq m^2t/k^2 )=1-m^2t/k^2(t+m_0)$.

The probability density $P(k) $can be obtained from $P(k)=
\frac{\partial P[k_i(t)<k]}{\partial k}$, which over long
time periods leads to the stationary solution:
\[P(k)=\frac{2m^2}{k^3}\]
giving $\gamma=3$, independent of $m$.

In the model, we assumed linear preferential attachment; that is,
$\prod (k)\sim k$


\section{待整理}
小世界网络

BA network:power law:

Preferential  attachment:
富人和富人成为朋友的概率
通过局部信息通过全局。

知道社会里有多少染病了,来判断局部


1、另一个方向:从社会资本的视角来看社会关系给自己带来的财富。
网络中心度意味着social capital,每个人都会拿自己的朋友数和自己朋
友的朋友的数量会做一个比较。

2、
自己的社会资本比我多数的朋友的社会资本少。
社会资本:首先需要很好的定义社会资本!然后进行比较!
每个人都想巴结有本事儿的,但是有本事的不想让更多人巴结自己。
现在的网络都没有考虑个体初始的一致性问题,都假设所有的nodes一
开始都是一样的。


3、
(1)、搜索模型search theory有没有和网络进行结合
\begin{itemize}
	\item Density, social networks and job search methods: Theory and application to Egypt.
	Journal of Development Economics Volume 78, Issue 2, December 2005, Pages 443-473
	
	\item The Impact of Informal Networks on Quit Behavior. Datcher, L. (1983). The impact of informal networks on quit behavior. The review of Economics and Statistics, 491-495.
	\item Good Friends in Bad Times? Social Networks and Job Search among the Unemployed in Sweden
	\item Cingano, F., \& Rosolia, A. (2012). People I know: job search and social networks. Journal of Labor Economics, 30(2), 291-332.
	\item Giulietti, C., Caliendo, M., Schmidl, R.,\& Uhlendorff, A. (2011). Social networks, job search methods and reservation wages: evidence for Germany. International Journal of Manpower.
	\item Identity and search in social networks. Watts, D. J., Dodds, P. S., \& Newman, M. E. (2002). Identity and search in social networks. science, 296(5571), 1302-1305.
\end{itemize}

(2)、和搜索模型和朋友悖论,$\text{search theory} $来解决复工问题。
不复工:解除概率低
复工:接触机会多,感染概率更大

(2.1):每个人都进行搜索,花费了成本。

付出的努力越高,感染概率更大,如果和网络结合。
可以解出来一个均衡的解!

完全信息背景下可以接触一个均衡解
从总体估计局部信息!
这样也可以求出一个均衡,看看会不会有偏差。

\subsection{灵感}
\begin{itemize}
	\item 把小世界网络和孤岛网络相结合!分析不同省市推送的相关信息如何影响人们决策!
	\item  由于网络的演化规律——穷者越穷,富者越富,社会资本分布非常不公平,在互联网时代,进一步加大了贫富差距!
	\item  友谊悖论是否影响个体和邻居的估计差异,依赖于degree的函数形式
\end{itemize}

可以从一些例子里面可以用来计算!
\paragraph{区域如何平衡发展:虹吸效应 }
一定要考虑到城市的虹吸效应!
\section{待求解}

\[E({d_i}) = \frac{E}{n} = \bar d\]

\[E({d_{{N_i}}}) = E({d_i}) + \frac{{Var({d_i})}}{{E({d_i})}}\]



\[{P_1} = {P_{i,t + 1}}({d_i}) = 1 - {\prod _{j \in {N_i}}}(1 - \frac{1}{n}{m_t}\alpha )\]

\[{P_2} = {P_{i,t + 1}}({d_i}) = 1 - {\prod _{j \in {N_i}}}(1 - {\beta _j}\alpha ) = 1 - {\prod _{j \in {N_i}}}(1 - \frac{{{d_j}}}{E}{m_t}\alpha )\]








\cite{zuckerman2001makes}

\cite{2001}
\bibliographystyle{plainnat}
\bibliography{ref}
\end{document} 